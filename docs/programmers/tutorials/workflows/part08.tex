\documentclass[english,serif,mathserif,xcolor=pdftex,dvipsnames,table]{beamer}
\usetheme{gc3}

\usepackage[T1]{fontenc}
\usepackage[utf8]{inputenc}
\usepackage{babel}

\usepackage{gc3}

\title[Sequencing tasks]{%
  Running tasks in sequence: \\
  \texttt{SequentialTaskCollection} and \texttt{StagedTaskCollection}
}
\author[R. Murri, S3IT UZH]{%
  Riccardo Murri \texttt{<riccardo.murri@uzh.ch>}
  \\[1ex]
  \emph{S3IT: Services and Support for Science IT}
  \\[1ex]
  University of Zurich
}
\date{Aug.~28 -- Sep.~1, 2017}


\begin{document}

% title frame
\maketitle


\begin{frame}[fragile]
  \frametitle{Basic use of SequentialTaskCollection}

\begin{python}
from gc3libs.workflow \
  import SequentialTaskCollection

class MySequence~\HL{(SequentialTaskCollection)}~:
  # ...
  def __init__(self, ...):
    app1 = FirstApp(...)
    app2 = SecondApp(...)
    SequentialTaskCollection.__init__(
      self, [app1, app2])
\end{python}

  \+
  A \texttt{SequentialTaskCollection} runs a list of tasks one at a time, in the order given.
\end{frame}


\begin{frame}[fragile]
  \frametitle{Basic use of SequentialTaskCollection}

\begin{python}
from gc3libs.workflow \
  import SequentialTaskCollection

class MySequence(SequentialTaskCollection):
  # ...
  def __init__(self, ...):
    app1 = FirstApp(...)
    app2 = SecondApp(...)
    SequentialTaskCollection.__init__(
      self, ~\HL{[app1, app2]}~)
\end{python}

  \+
  Initialize a \texttt{SequentialTaskCollection} \\
  with a list of tasks to run.
\end{frame}


\begin{frame}[fragile]
  \frametitle{Running tasks in sequence}

\begin{python}
class MyScript(SessionBasedScript):
  # ...
  def new_tasks(self, extra):
    tasks_to_run = [
      MySequence(...)
    ]
    return tasks_to_run
\end{python}

  \+ You can then run the entire sequence by returning it from
  \lstinline|new_tasks()|.
\end{frame}


\begin{frame}
  \begin{exercise*}[8.A]

    Write a \texttt{priceplot.py} script that performs the following two steps:
    \begin{enumerate}
    \item Run the
      \href{https://github.com/uzh/gc3pie/blob/master/docs/programmers/tutorials/workflows/downloads/simAsset.R}{\texttt{simAsset.R}}
      script (from Exercise~6.D) with the parameters given on the command line,
      and
    \item Feed the \texttt{results.csv} file it outputs into the
      \href{https://github.com/uzh/gc3pie/blob/master/docs/programmers/tutorials/workflows/downloads/saplot.py}{\texttt{saplot.py}}
      script and retrieve the produced \texttt{saplot.pdf} file.
    \end{enumerate}

    \+
    Run it like \texttt{simAsset.R}, for example:
\begin{semiverbatim}
  \$ python priceplot.py 50 0.04 0.1 0.27 10 40
\end{semiverbatim}

  \end{exercise*}
\end{frame}


\begin{frame}[fragile]
  \frametitle{Running jobs in sequence}

  \texttt{StagedTaskCollection} provides a simple interface for
  constructing sequences of tasks, but only when the number and
  content of steps is \emph{known and fixed} at programming time.

  \+
  (By contrast, the most general \texttt{SequentialTaskCollection}
  can alter the sequence on the fly, insert new stages while running
  and loop back. But the code is also harder to write.)
\end{frame}


\begin{frame}[fragile]
  \begin{columns}[t]
    \begin{column}{0.6\textwidth}
\begin{lstlisting}[basicstyle=\footnotesize\ttfamily]
class Pipeline~\HL{(StagedTaskCollection)}~:
  def __init__(self, image):
    super(Pipeline).__init__(self)
    self.source = image

  def stage0(self):
    # ...

  def stage1(self):
    # ...

  # ...
  def stage~{\bfseries\itshape X}~(self):
    # ...
      \end{lstlisting}
    \end{column}
    \begin{column}{0.4\textwidth}
      \raggedleft

      \+\+\small
      Example: \\
      subclassing a
      \texttt{StagedTaskCollection}
    \end{column}
  \end{columns}
\end{frame}


\begin{frame}[fragile]
  \begin{columns}[c]
    \begin{column}{0.6\textwidth}
      \begin{lstlisting}[basicstyle=\footnotesize\ttfamily]
class Pipeline(StagedTaskCollection):
  def __init__(self, image):
    super(Pipeline).__init__(self)
    self.source = image

  def ~\HL{stage0(self)}~:
    # ...

  def ~\HL{stage1(self)}~:
    # ...

  # ...
  def ~\HL{stage{\bfseries\itshape X}(self)}~:
    # ...
      \end{lstlisting}
    \end{column}
    \begin{column}{0.4\textwidth}
      \raggedleft

      Stages are numbered starting from $0$.

      \+
      You can have as many stages as you want.
    \end{column}
  \end{columns}
\end{frame}


\begin{frame}[fragile]
  \begin{columns}[c]
    \begin{column}{0.6\textwidth}
      \begin{lstlisting}[basicstyle=\footnotesize\ttfamily]
class Pipeline(StagedTaskCollection):
  # ...

  def stage0(self):
    # run 1st step
    ~\HL{return Application}~(
      ['convert', self.source,
       '-colorspace', 'gray',
       'grayscale_' + self.source],
      inputs = [self.source],
      ...)

  # ...
      \end{lstlisting}
    \end{column}
    \begin{column}{0.4\textwidth}
      \raggedleft

      Each \texttt{stage{\itshape X}} method can return a \texttt{Task}
      instance, that will run as the $X$-th step in the sequence.
    \end{column}
  \end{columns}
\end{frame}


\begin{frame}[fragile]
  \begin{columns}[c]
    \begin{column}{0.6\textwidth}
      \begin{lstlisting}[basicstyle=\footnotesize\ttfamily]
class Pipeline(StagedTaskCollection):
  # ...

  def stage1(self):
    ~\HL{\textbf{if} \textbf{\color{gray} self}.tasks[0].execution.exitcode != 0:}~
      # bail out
      return (0, 1)
    else:
      # run 2nd step
      return Application(...)

  # ...
  def stage~{\bfseries\itshape X}~(self):
    # ...
      \end{lstlisting}
    \end{column}
    \begin{column}{0.4\textwidth}
      \raggedleft

      \+\+\+\+\+
      In later stages you can check the exit code of
      earlier ones, and decide whether to continue the sequence or
      abort.
    \end{column}
  \end{columns}
\end{frame}


\begin{frame}[fragile]
  \begin{columns}[c]
    \begin{column}{0.6\textwidth}
      \begin{lstlisting}[basicstyle=\footnotesize\ttfamily]
class Pipeline(StagedTaskCollection):
  # ...

  def stage1(self):
    if ...:
      # bail out
      ~\HL{return (0, 1)}~
    else:
      # run 2nd step
      return Application(...)

  # ...
  def stage~{\bfseries\itshape X}~(self):
    # ...
      \end{lstlisting}
    \end{column}
    \begin{column}{0.4\textwidth}
      \raggedleft

      \+\+\+\+\+
      To abort the sequence, return an integer (termination
      status) or a pair \emph{(signal, exit code)}, instead of a
      \texttt{Task} instance.

      \+
      This sets the collections' own signal and exit code, and also sets the
      state as \texttt{TERMINATED}.
    \end{column}
  \end{columns}
\end{frame}


\begin{frame}[fragile]
  \frametitle{Detour: BLAST, again}

  Another use of the BLAST tool is to search for given ``query''
  proteins in a data base.  Large curated DBs are available, but one
  may want to build a custom DB.

  \+
  Building a DB from a set of FASTA-format files \texttt{p1.faa}
  \texttt{p2.faa} and \texttt{p3.faa}, and querying it is a 3-step
  process:
\begin{sh}
  cat p1.faa p2.faa p3.faa > db.faa
  formatdb -i db.faa
  blastpgp -i q.faa -d db.faa -e ...
\end{sh}

  \+ The \texttt{formatdb} step produces output files
  \texttt{db.faa.phr}, \texttt{db.faa.pin}, and \texttt{db.faa.psq}; all
  these files are \emph{inputs} to the \texttt{blastpgp} program.
\end{frame}


\begin{frame}[fragile]
  \begin{exercise*}[8.B] \emph{(Difficult)}

    Write a \texttt{blastdb.py} script to build a BLAST DB and query it.

    \+
    The \texttt{blastdb.py} script shall be invoked like this:
\begin{sh}
$ python blastdb.py query.faa p1.faa [p2.faa ...]
\end{sh}%$
    where arguments \texttt{new.faa}, \texttt{p1.faa}, etc. are FASTA-format files.

    \+
    The script should build a BLAST DB out of the files {p$N$.faa}.
    Then, it should query this database for occurrences of the
    proteins in \texttt{query.faa} using \texttt{blastpgp}.
  \end{exercise*}
\end{frame}


\begin{frame}
  \begin{exercise*}[8.C]
    Find out by running the \texttt{blastdb.py} script of Ex.~8.B:

    \+
    \begin{enumerate}
    \item What happens if an intermediate step fails and does not
      produce complete output?

      \+
    \item After the whole sequence turns to TERMINATED state, what is
      the value of its signal and exitcode?
    \end{enumerate}
  \end{exercise*}

  \+
  \begin{exercise*}[8.D]
    Implement (in \texttt{blastdb.py}) a ``cleanup'' feature that removes
    intermediate results (e.g., the ``\texttt{.phr}'' files) and only keeps the
    output from \texttt{blastpgp} \emph{if the whole sequence was successfully
      executed}.
  \end{exercise*}
\end{frame}

\end{document}

%%% Local Variables:
%%% mode: latex
%%% TeX-master: t
%%% End:
