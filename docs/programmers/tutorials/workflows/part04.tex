\documentclass[english,serif,mathserif,xcolor=pdftex,dvipsnames,table]{beamer}
\usetheme{gc3}

\usepackage[T1]{fontenc}
\usepackage[utf8]{inputenc}
\usepackage{babel}

\usepackage{gc3}

\title[Introduction]{%
  Customizing \\ session-based scripts
}
\author[R. Murri, S3IT UZH]{%
  Riccardo Murri \texttt{<riccardo.murri@uzh.ch>}
  \\[1ex]
  \emph{S3IT: Services and Support for Science IT}
  \\[1ex]
  University of Zurich
}
\date{Aug.~28 -- Sep.~1, 2017}

\begin{document}

% title frame
\maketitle


\begin{frame}
  \frametitle{Outline}
  \begin{enumerate}
  \item Customize positional argument processing
  \item Customize option processing
  \item More complex code in \texttt{new\_tasks()}
  \end{enumerate}
\end{frame}


\begin{frame}[fragile]
  \frametitle{What is command-line argument processing?}

  Problem:
\begin{sh}
$ python ex4a.py --foo=1 bar baz
\end{sh}%$

  \+
  How do we translate the above shell command invocation into a
  Python procedure call \\ \texttt{ex4a('bar', 'baz', foo=1)}?

  \+
  In other words: how do we access the command-line arguments given
  to the shell from Python code?
\end{frame}


\begin{frame}
  \frametitle{Command-line argument processing in GC3Pie}

  GC3Pie scripts use the standard Python
  library~\href{https://docs.python.org/2/howto/argparse.html}{\texttt{argparse}}.

  \+
  However, differently from \texttt{argparse}, GC3Pie scripts use
  separate ways to configure processing of options and positional
  arguments.
\end{frame}

\begin{frame}[fragile]
  \frametitle{Positional argument processing}

  Positional arguments are defined in a method called
  \lstinline|setup_args()|; override it in derived classes to change
  what arguments are accepted, their names and number.

  \+
  This is the default implementation:
\begin{python}
  def setup_args(self):
    self.add_param(
      'args', nargs='*', metavar='INPUT',
      help="Path to input file or directory.")
\end{python}
\end{frame}


\begin{frame}[fragile]
  \frametitle{Positional argument processing}

\begin{python}
  def setup_args(self):
    self.add_param(
      ~\HL{'args'}~, nargs='*', metavar='INPUT',
      help="Path to input file or directory.")
\end{python}

  \+
  The value of this command-line argument will be recorded in the
  variable \texttt{self.params.\HL{args}} in Python code.
\end{frame}


\begin{frame}[fragile]
  \frametitle{Positional argument processing: \texttt{nargs}}

\begin{python}
  def setup_args(self):
    self.add_param(
      'args', ~\HL{nargs='*'}~, metavar='INPUT',
      help="Path to input file or directory.")
\end{python}

  \+
  There are 0 or more arguments of this kind.  \\
  Makes \lstinline|self.params.args| into a Python list.
\end{frame}


\begin{frame}[fragile]
  \frametitle{Positional argument processing: \texttt{nargs}}

\begin{python}
  def setup_args(self):
    self.add_param('args', ~\HL{nargs=\ldots}~, ~{\em [\ldots]}~)
\end{python}

  \+
  Other possible values for the \texttt{nargs} parameter are:
  \begin{itemize}
  \item \lstinline|nargs='+'| --- there are \emph{1 or more} arguments,
    i.e., at least one argument is required.
  \item \lstinline|nargs=|$N$ --- there are exactly $N \geq 1$ arguments of this kind.
  \item \lstinline|nargs=None| (default) --- one single mandatory
    argument of this kind; \textbf{only in this case
      \lstinline|self.params.args| is a \emph{string}, not a list.}
  \end{itemize}
\end{frame}


\begin{frame}[fragile]
  \frametitle{Positional argument processing: \texttt{metavar}}

\begin{python}
  def setup_args(self):
    self.add_param(
      'args', nargs='*', ~\HL{metavar='INPUT'}~,
      help="Path to input file or directory.")
\end{python}

  \+
  The name of the command-line argument as displayed to users in
  usage and help texts:
\begin{sh}
$ python solutions/ex2c.py --help
usage: ex2c [-h] [-V] [-v] [--config-files CONFIG_FILES] [-c NUM]
            [-m GIGABYTES] [-r NAME] [-w DURATION] [-s PATH] [-u URL] [-N]
            [-C NUM] [-J NUM] [-o DIRECTORY] [-l [STATES]]
            ~\HL{[INPUT [INPUT ...]]}~
\end{sh}%$

  \+
  If not given, defaults to the uppercased version of the ``internal'' name.
\end{frame}


\begin{frame}[fragile]
  \frametitle{Positional argument processing: \texttt{help}}

\begin{python}
  def setup_args(self):
    self.add_param(
      'args', nargs='*', metavar='INPUT',
      ~\HL{help="Path to input file or directory.")}~
\end{python}

  \+
  Description of the command-line argument in \texttt{-{}-help} text.

\begin{sh}
$ python solutions/ex2c.py --help
usage: ex2c [-h] ~{\em [\ldots]}~

positional arguments:
  ~\HL{INPUT}~         ~\HL{Path to input file or directory.}~
\end{sh}%$
\end{frame}


\begin{frame}[fragile]
  \frametitle{Multiple positional argument processing}

  Calls to \lstinline|self.add_param()| can be repeated to
  parse many different command-line arguments in sequence:
\begin{python}
class AScript(SessionBasedScript):
  # ~\em [\ldots]~
  def setup_args(self):
    self.add_param('infile', help="Input file")
    self.add_param('radius', help="Convolution radius")
    self.add_param('sigma',  help="Threshold")
\end{python}

  \+
  With the above definition, the following command-line:
\begin{sh}
  $ python example.py file.img 10 2.1
\end{sh}%$
  generates the equivalent of the following Python code:
  \begin{python}
  self.params.infile = 'example.py'
  self.params.radius = '10'  # it's a string!
  self.params.sigma = '2.1'  # this one too!
  \end{python}

\end{frame}


\begin{frame}[fragile]
  \frametitle{Detour: From grayscale to colors}
\begin{sh}
$ convert gray-bfly.jpg \
    ( xc:blue xc:magenta xc:yellow +append ) \
    -clut color-bfly.jpg
\end{sh}%$
\begin{tabular}{ccc}
  {\includegraphics[width=0.4\linewidth]{fig/gray-butterfly.jpg}}
  &
  {\includegraphics[width=0.1\linewidth,totalheight=0.25\textheight]{fig/arrow.pdf}}
  &
  {\includegraphics[width=0.4\linewidth]{fig/rgb-butterfly.jpg}}
\end{tabular}
\end{frame}


\begin{frame}[fragile]
  \begin{exercise*}[4.A]
    Write a \texttt{colorize.py} script to apply this colorization process to a set of grayscale images.

    \+
    The \texttt{colorize.py} script shall be invoked like this:
\begin{sh}
$ python colorize.py ~\em c1~ ~\em c2~ ~\em c3~ ~\em img1~ ~[\em img2 \ldots]~
\end{sh}%$
    where \texttt{\em c1}, \texttt{\em c2}, \texttt{\em c3} are color
    names and \texttt{\em img1}, \texttt{\em img2} are image files.

    \+
    Each image shall be processed in a separate colorization task.
  \end{exercise*}
\end{frame}


\begin{frame}[fragile]
  \frametitle{Argument types}

  You can ask
  \href{https://docs.python.org/2/howto/argparse.html}{\texttt{argparse}}
  and GC3Pie to convert a command-line argument to a certain Python type.

  \+ For example:
\begin{python}
class AScript(SessionBasedScript):
  # ~\em [\ldots]~
  def setup_args(self):

    # this argument is a string (default type)
    self.add_param('infile', type=str,   help="...")

    # the `radius` argument is an integer
    self.add_param('radius', type=int,   help="...")

    # the `sigma` argument is a floating-point number
    self.add_param('sigma',  type=float, help="...")
\end{python}
\end{frame}


\begin{frame}[fragile]
  \frametitle{Argument types}

  \textbf{Declaring argument types makes for better usability!}

  \+ If an argument does not match its type, \\ the script exists immediately
  and the user is notified with a clear error message.

  \+
\begin{sh}
$ python downloads/argp.py foo.txt 123 x
usage: argparse [-h] ~\em [\ldots]~
                infile radius sigma
argp: error: argument sigma: invalid float value: 'x'
\end{sh}%$
\end{frame}


\begin{frame}[fragile]
  \frametitle{Argument types}\small

  In addition to the standard Python types, GC3Pie provides other
  validation functions to ensure arguments meet commonly-found
  conditions.

\begin{python}
from gc3libs.cmdline import \
  existing_file, positive_int

class AScript(SessionBasedScript):
  # ~\em [\ldots]~
  def setup_args(self):
    # reject non-existent input files outright
    self.add_param('infile', type=existing_file, ~\ldots~)
    # force radius to be > 0
    self.add_param('radius', type=positive_int, ~\ldots~)
    #  sigma is a floating-point number
    self.add_param('sigma',  type=float, ~\ldots~)
\end{python}

\begin{references}
  \tiny\url{http://gc3pie.readthedocs.io/en/master/programmers/api/gc3libs/cmdline.html}
\end{references}
\end{frame}


\begin{frame}[fragile]
  \frametitle{Command-line option processing}

  Command-line options can be \textbf{added} to a session-based script
  by overriding the \lstinline|setup_options()| method:
  \begin{python}
class AScript(SessionBasedScript):
  # ~\em [\ldots]~
  def setup_options(self):
    self.add_param('--e-value', '-e', dest='e_value',
                   type=float, default=10.0,
                   help="Expectation value")
  \end{python}
\end{frame}


\begin{frame}[fragile]
  \frametitle{Command-line option processing}

  \begin{python}
class AScript(SessionBasedScript):
  # ~\em [\ldots]~
  def setup_options(self):
    self.add_param('--e-value', ~\HL{'-e'}~, dest='e_value',
                   type=float, default=10.0,
                   help="Expectation value")
  \end{python}

  \+
  Aliases and abbreviations for options can be defined.
\end{frame}


\begin{frame}[fragile]
  \frametitle{Command-line option processing}

  \begin{python}
class AScript(SessionBasedScript):
  # ~\em [\ldots]~
  def setup_options(self):
    self.add_param('--e-value', '-e', ~\HL{dest='e\_value'}~,
                   type=float, default=10.0,
                   help="Expectation value")
  \end{python}

  \+
  The value of this option will be stored in \lstinline|self.params.|\texttt{\HL{e\_value}}.

\end{frame}


\begin{frame}[fragile]
  \frametitle{Command-line option processing}

  \begin{python}
class AScript(SessionBasedScript):
  # ~\em [\ldots]~
  def setup_options(self):
    self.add_param('--e-value', '-e', dest='e_value',
                   ~\HL{type=float}~, default=10.0,
                   help="Expectation value")
  \end{python}

  \+
  Types work exactly as for positional arguments.
\end{frame}


\begin{frame}[fragile]
  \frametitle{Command-line option processing}

  \begin{python}
class AScript(SessionBasedScript):
  # ~\em [\ldots]~
  def setup_options(self):
    self.add_param('--e-value', '-e', dest='e_value',
                   type=float, ~\HL{default=10.0}~,
                   help="Expectation value")
  \end{python}

  \+
  If the option is not present on the command-line, the associated
  Python variable (here,
  \lstinline|self.params.|\texttt{\HL{e\_value}}) takes this value.
\end{frame}


\begin{frame}[fragile]
  \frametitle{Detour: BLAST}

  BLAST is a suite of programs to perform search and alignment of
  nucleotides and proteins.

  \+ One common use of BLAST is the following: given a file describing
  a new organism, compare it one-to-one to a set of known organisms to
  find similarities.

  \+ The command-line invocation for one such comparison would look like this:
\begin{sh}
$ blastp -query new.faa -subject known.faa \
  -evalue 1e-6 -outfmt 5
\end{sh}%$

  \+
  \begin{center}
    \tiny\itshape (Thanks to Lars Malmstroem for suggesting the following BLAST-related exercises.)
  \end{center}
\end{frame}

\begin{frame}[fragile]
  \begin{exercise*}[4.B]\small
    Write a \texttt{topblast.py} script to perform 1-1 BLAST comparisons.

    \+
    The \texttt{topblast.py} script shall be invoked like this:
\begin{sh}
$ python topblast.py [-e ~$T$~] [-m ~$F$~] \
    new.faa k1.faa [k2.faa ...]
\end{sh}%$
    where:
    \begin{itemize}
    \item Option \texttt{-e} (alias: \texttt{-{}-e-value}) takes a floating
      point threshold argument $T$ (defaulting to $1.0$);
    \item Option \texttt{-m} (alias: \texttt{-{}-output-format}) takes a
      single-digit integer argument $F$ (default $5$);
    \item Arguments \texttt{new.faa}, \texttt{k1.faa}, etc. are files.
    \end{itemize}

    \+ The script should generate and run comparisons between
    \texttt{new.faa} and each of the \texttt{k1.faa}, \texttt{k2.faa},
    ..., etc.  Each 1-1 comparison should run as a separate task.  All
    of them share the same settings for the \texttt{-evalue} and
    \texttt{-outfmt} options for \texttt{blastp}.
  \end{exercise*}
\end{frame}


\begin{frame}[fragile]
  \begin{exercise*}[4.C] \emph{(Homework)}

    Modify the \texttt{topblast.py} script that you've written in Exercise 4.B
    to be invoked like this:
\begin{sh}
$ python topblast.py [-e ~$T$~] [-m ~$F$~] new.faa dir
\end{sh}%$

    \+ Input files describing the ``known'' subjects should be found by
    recursively scanning the given directory path.

    \+ Bonus points if the modified script exists with a correct error
    message in case \texttt{new.faa} is not an existing file, or
    \texttt{dir} is not a valid directory path.
  \end{exercise*}
\end{frame}

\end{document}

%%% Local Variables:
%%% mode: latex
%%% TeX-master: t
%%% End:
