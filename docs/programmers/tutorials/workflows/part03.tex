\documentclass[english,serif,mathserif,xcolor=pdftex,dvipsnames,table]{beamer}
\usetheme{gc3}

\usepackage[T1]{fontenc}
\usepackage[utf8]{inputenc}
\usepackage{babel}

\usepackage{gc3}

\title[Debugging commands]{%
  Useful \\ debugging commands
}
\author[R. Murri, S3IT UZH]{%
  Riccardo Murri \texttt{<riccardo.murri@uzh.ch>}
  \\[1ex]
  \emph{S3IT: Services and Support for Science IT}
  \\[1ex]
  University of Zurich
}
\date{January~23--27, 2017}

\begin{document}

% title frame
\maketitle


\part{Recap of session-based scripts}


\begin{frame}
  \frametitle{Session-based scripts}

  All the scripts we've seen so far are \emph{session-based scripts}.

  \+ A \emph{session} is just a named collection of jobs.

  \+ A \emph{session-based script} creates a session and runs all the
  tasks in it until completion.
\end{frame}


\begin{frame}
  \frametitle{Create a session}

  A session-based script \alert{creates a session}
  \\
  and runs all the tasks in it until completion.

  \+ Create session \texttt{S}:
\begin{semiverbatim}
    \$ ./warholize.py bfly.jpg -{}-session S
\end{semiverbatim}
\end{frame}


\begin{frame}[fragile]
  \frametitle{Run a session until done}

  A session-based script creates a session
  \\
  and \alert<1>{runs all the jobs in it until completion.}

  \+ Run jobs in session \texttt{S},
  polling for updates every 5 seconds:
\begin{semiverbatim}\small
    \$ ./warholize.py bfly.jpg -{}-session S -{}-watch 5
\end{semiverbatim}

  \+ \uncover<2>{
    \alert<2>{You can stop a GC3Pie script by pressing \emph{Ctrl+C}. \\
    Run it again to resume activity from where it stopped.}
  }
\end{frame}


\part{Inspecting sessions}

\begin{frame}
  \frametitle{Alternate display of session contents, I}

  Display top-level tasks in session \texttt{S}:
\begin{semiverbatim}
    \$ gsession list S
\end{semiverbatim}%$

  \begin{exercise*}[3.A]
    Now try this yourself.
  \end{exercise*}
\end{frame}


\begin{frame}[fragile]
  \frametitle{WTF??}
  \begin{stdout}
> gsession list S
gc3.gc3libs: WARNING: Failed loading file '/home/ubuntu/S/jobs/WarholizeWorkflow.108': ImportError: No module named warholize
  ^\ldots^
LoadError: Failed retrieving object from file '/home/ubuntu/S/jobs/WarholizeWorkflow.108': ImportError: No module named warholize
gc3.gc3libs: WARNING: Ignoring error from loading 'ParallelTaskCollection.107': Failed retrieving object from file '/home/ubuntu/S/jobs/ParallelTaskCollection.107': LoadError: Failed retrieving object from file '/home/ubuntu/S/jobs/WarholizeWorkflow.108': ImportError: No module named warholize
+-------+----------+-------+------+
| JobID | Job name | State | Info |
+-------+----------+-------+------+
+-------+----------+-------+------+
  \end{stdout}

  \pause
  \+ In order to work, all GC3Pie utilities need to access the Python
  script that generated the tasks and session.

  \+ To fix: set the \lstinline|PYTHONPATH| variable to the directory
    containing your script:
    \begin{lstlisting}[language=sh]
$ export PYTHONPATH=$PWD
    \end{lstlisting}%$
\end{frame}


\begin{frame}
  \frametitle{Alternate display of session contents, II}

  Display \emph{all} tasks in session \texttt{S}:
\begin{semiverbatim}
    \$ gsession list -{}-recursive S
\end{semiverbatim}%$
\end{frame}


\begin{frame}
  \frametitle{Alternate display of session contents, III}

  Display summary of tasks in session \texttt{S}:
\begin{semiverbatim}
    \$ gstat -b -s S
\end{semiverbatim}
\end{frame}


\begin{frame}[fragile]
  \frametitle{Show log of actions in session}

  The \texttt{gession log} command prints out the sequence of actions
  and state changes for all tasks in a session:

\begin{stdout}
$ ^\HL{gsession log ex2b}^
Jul 09 22:13:53 GrayscaleApp.6: Submitting to 'localhost'
Jul 09 22:13:53 GrayscaleApp.6: SUBMITTED
Jul 09 22:13:53 GrayscaleApp.6: Submitted to 'localhost'
Jul 09 22:13:58 GrayscaleApp.6: TERMINATING
Jul 09 22:13:58 GrayscaleApp.6: Final output downloaded to 'grayscale.d'
Jul 09 22:13:58 GrayscaleApp.6: TERMINATED
\end{stdout}%$
\end{frame}


\begin{frame}[fragile]
  \frametitle{Dump contents of a task}

  The \texttt{ginfo} command allows you to dump the contents of a
  specific task, or all tasks in a session:

\begin{stdout}[basicstyle=\tiny\ttfamily]
$ ^\HL{ginfo -v -s ex2b}^
GrayscaleApp.6
  arguments: convert, bfly.jpg, -colorspace, gray, gray-bfly.jpg
  ^\em [\ldots]^
  execution:
    ^\em [\ldots]^
    history:
      - Submitting to 'localhost' at Sun Jul 10 00:27:02 2016
      ^\em [\ldots]^
    lrms_execdir: /home/gc3pie/docs/programmers/tutorials/workflows/gc3libs.SMGNxr
    lrms_jobid: 23183
    resource_name: localhost
    ^\em [\ldots]^
  inputs:
    file:///home/gc3pie/docs/programmers/tutorials/workflows/bfly.jpg: bfly.jpg
  ^\em [\ldots]^
  output_dir: grayscale.d
  outputs:
    gray-lbfly.jpg: file, , gray-bfly.jpg, None, None, , None, None
    stderr.txt: file, , stderr.txt, None, None, , None, None
    stdout.txt: file, , stdout.txt, None, None, , None, None
  ^\em [\ldots]^
  stderr: stderr.txt
  stdin: None
  stdout: stdout.txt
  ^\em [\ldots]^
\end{stdout}%$
\end{frame}


\begin{frame}[fragile]
  \frametitle{Dump contents of a task, II}

  Note that \textbf{without the \texttt{-v}} option, \texttt{ginfo}
  limits its output to the \texttt{.execution} attribute of a
  \texttt{Task}/\texttt{Application} object:

\begin{stdout}[basicstyle=\tiny\ttfamily]
$ ^\HL{ginfo -s S}^
GrayscaleApp.6
  ^\em [\ldots]^
  history:
    - Submitting to 'localhost' at Sun Jul 10 00:27:02 2016
    ^\em [\ldots]^
  lrms_execdir: /home/gc3pie/docs/programmers/tutorials/workflows/gc3libs.SMGNxr
  lrms_jobid: 23183
  resource_name: localhost
  ^\em [\ldots]^
\end{stdout}%$
\end{frame}


\part{Session management}

\begin{frame}[fragile]
  \frametitle{Abort all tasks in a session}

  The \texttt{gsession abort} command kills all tasks in a session.
\begin{stdout}
$ ^\HL{gsession abort S}^
\end{stdout}%$

  \+
  It produces normally no output; use the \texttt{-v} option to see a
  log of actions taken.

  \+ \small \alert{\textbf{Note:} it is important that sessions are
  terminated!}  Otherwise, GC3Pie will consider part of the resources
  as still allocated to a task.  This is especially evident when
  running on a single computer: after launching a few tasks, GC3Pie
  will stop and refuse to run anything.
\end{frame}


\begin{frame}[fragile]
  \frametitle{Manual cleanup of the {\ttfamily localhost} resource}

  In case of incomplete cleanup, you will still run into this error:
\begin{stdout}
  ERROR: Resource localhost already running maximum allowed number of jobs
\end{stdout}

  \+
  As a last resort, you can inspect directory \lstinline|$HOME/.gc3/shellcmd.d|:
\begin{stdout}
  $ ls $HOME/.gc3/shellcmd.d
  15843
\end{stdout}
  Each of these files is an allocated execution slot in the \texttt{localhost} resource.
  Delete the files to free up the slot.

  \+ (The file name is the PID of the process, in case you want to
  check if a command is still running before you make GC3Pie forget
  about it.)
\end{frame}


\begin{frame}
  \frametitle{Aborting a single task}

  To stop and abort a single task, use the \texttt{gkill} command:
\begin{semiverbatim}
    \$ gkill -s S MyApplication.123
\end{semiverbatim}

\end{frame}


\part{Combining commands}

\begin{frame}[fragile]
  \frametitle{Selecting tasks from a session, I}

  The \texttt{gselect} command is the go-to tool for selective listing
  of tasks in a session.  For example, to list finished tasks:
\begin{semiverbatim}
    \$ gselect -s S -{}-state TERMINATED
\end{semiverbatim}

  \+ The output of \texttt{gselect} is a list of task IDs, to be fed
  into another GC3Pie command.  For example, to kill all queued tasks:
  \begin{stdout}
    > gselect -s S --state SUBMITTED | xargs gkill -s S
  \end{stdout}
\end{frame}

\begin{frame}[fragile]
  \frametitle{Selecting tasks from a session, II}

  The \texttt{gselect} command has many different \\ options to select tasks:
  \begin{stdout}
> gselect --help
usage: gselect [-h] [-V] [-v] [--config-files CONFIG_FILES] -s SESSION
               [--error-message REGEXP] [--input-file FILENAME]
               [--jobname REGEXP] [--jobid REGEXP] [-l STATES]
               [--output-file FILENAME] [--output-message REGEXP]
               [--successful] [--submitted-after DATE]
               [--submitted-before DATE] [--unsuccessful]
  \end{stdout}

  \+
  \begin{exercise*}[3.B]
    Use \texttt{gselect} to print the IDs of the ``TricolorizeImage''
    tasks in the last ``Warholize'' session.
  \end{exercise*}
\end{frame}

\end{document}

%%% Local Variables:
%%% mode: latex
%%% TeX-master: t
%%% End:
