\documentclass[english,serif,mathserif,xcolor=pdftex,dvipsnames,table]{beamer}
\usetheme{gc3}

\usepackage[T1]{fontenc}
\usepackage[utf8]{inputenc}
\usepackage{babel}

\usepackage{gc3}

\title[Post-processing]{%
  Application control and post-processing
}
\author[R. Murri, S3IT UZH]{%
  Riccardo Murri \texttt{<riccardo.murri@uzh.ch>}
  \\[1ex]
  \emph{S3IT: Services and Support for Science IT}
  \\[1ex]
  University of Zurich
}
\date{Aug.~28 -- Sep.~1, 2017}

\begin{document}

% title frame
\maketitle


\section{Application run states}
\part{Application run states}

\begin{frame}[fragile]
\frametitle{Application lifecycle}

\begin{columns}[c]
  \begin{column}{0.5\textwidth}
\begin{lstlisting}[basicstyle=\footnotesize\ttfamily,keywordstyle=\normalfont]
$ ./grayscale.py bfly.jpg
~\em [\ldots]~
         NEW  1/1  (100.0%)
     RUNNING  0/1   (0.0%)
     STOPPED  0/1   (0.0%)
   SUBMITTED  0/1   (0.0%)
  TERMINATED  0/1   (0.0%)
 TERMINATING  0/1   (0.0%)
     UNKNOWN  0/1   (0.0%)
       total  1/1  (100.0%)
\end{lstlisting}%$
  \end{column}
  \begin{column}{0.5\textwidth}
    \raggedleft
    \texttt{Application} objects can be in one of several states.

    \+
    (A session-based script prints a table of all managed applications and their states.)
  \end{column}
\end{columns}

\+
\begin{columns}[c]
  \begin{column}{0.5\textwidth}
\begin{lstlisting}[basicstyle=\footnotesize\ttfamily]
>>> print(app.execution.state)
'TERMINATED'
\end{lstlisting}
  \end{column}
  \begin{column}{0.5\textwidth}
    \raggedleft
    The current state is stored in the \texttt{.execution.state} instance attribute.
  \end{column}
\end{columns}

\+
\begin{references}
  \tiny
  \url{http://gc3pie.readthedocs.io/en/master/programmers/api/gc3libs.html#gc3libs.Run.state}
\end{references}
\end{frame}


\begin{frame}[fragile]
\frametitle{Application lifecycle: state NEW}

\begin{columns}[c]
  \begin{column}{0.5\textwidth}
    \includegraphics[height=0.7\textheight]{fig/states-NEW}
  \end{column}
  \begin{column}{0.5\textwidth}
    \raggedleft
    \textbf{NEW} is the state of ``just created'' Application objects.

    \+
    The Application has not yet been sent off to a compute
    resource: it only exists locally.
  \end{column}
\end{columns}
\end{frame}


\begin{frame}[fragile]
\frametitle{Application lifecycle: state SUBMITTED}

\begin{columns}[c]
  \begin{column}{0.5\textwidth}
    \includegraphics[height=0.7\textheight]{fig/states-SUBMITTED}
  \end{column}
  \begin{column}{0.5\textwidth}
    \raggedleft
    \emph{SUBMITTED} applications have been successfully sent to a
    computational resource.

    \+
    (The transition to \emph{RUNNING} happens automatically, as we
    do not control the remote execution.)
  \end{column}
\end{columns}
\end{frame}


\begin{frame}[fragile]
\frametitle{Application lifecycle: state RUNNING}

\begin{columns}[c]
  \begin{column}{0.5\textwidth}
    \includegraphics[height=0.7\textheight]{fig/states-RUNNING}
  \end{column}
  \begin{column}{0.5\textwidth}
    \raggedleft
    \emph{RUNNING} state happens when the computational job associated to an
    application starts executing on the computational resource.
  \end{column}
\end{columns}
\end{frame}


\begin{frame}[fragile]
\frametitle{Application lifecycle: state STOPPED}

\begin{columns}[c]
  \begin{column}{0.5\textwidth}
    \includegraphics[height=0.7\textheight]{fig/states-STOPPED}
  \end{column}
  \begin{column}{0.5\textwidth}
    \raggedleft
    A task is in \emph{STOPPED} state when its execution has been
    blocked at the remote site and GC3Pie cannot recover
    automatically.

    \+
    User or sysadmin intervention is required for a task to get out
    of \emph{STOPPED} state.
  \end{column}
\end{columns}
\end{frame}


\begin{frame}[fragile]
\frametitle{Application lifecycle: state UNKNOWN}

\begin{columns}[c]
  \begin{column}{0.5\textwidth}
    \includegraphics[height=0.7\textheight]{fig/states-UNKNOWN}
  \end{column}
  \begin{column}{0.5\textwidth}
    \raggedleft
    A task is in \emph{UNKNOWN} state when GC3Pie can no
    longer monitor it at the remote site.

    \+
    (As this might be due to network failures, jobs \emph{can} get
    out of \emph{UNKNOWN} automatically.)
  \end{column}
\end{columns}
\end{frame}


\begin{frame}[fragile]
\frametitle{Application lifecycle: state TERMINATING}

\begin{columns}[c]
  \begin{column}{0.5\textwidth}
    \includegraphics[height=0.7\textheight]{fig/states-TERMINATING}
  \end{column}
  \begin{column}{0.5\textwidth}
    \raggedleft
    \emph{TERMINATING} state when a computational job has finished
    running, for whatever reason.

    \+
    (Transition to \emph{TERMINATED} only happens when \texttt{fetch\_output} is called.)
  \end{column}
\end{columns}
\end{frame}


\begin{frame}[fragile]
\frametitle{Application lifecycle: state TERMINATED}

\begin{columns}[c]
  \begin{column}{0.5\textwidth}
    \includegraphics[height=0.7\textheight]{fig/states-TERMINATED}
  \end{column}
  \begin{column}{0.5\textwidth}
    \raggedleft
    A job is \emph{TERMINATED} when its final output has been
    retrieved and is available locally.

    \+
    The exit code of \emph{TERMINATED} jobs can be inspected to
    find out whether the termination was successful or unsuccessful,
    or if the program was forcibly ended.
  \end{column}
\end{columns}
\end{frame}


\section{Post-processing}
\part{Post-processing}

\begin{frame}[fragile]
  \frametitle{Post-processing features, I}

  When the remote computation is done, the \texttt{terminated} method
  of the application instance is called.

  \+
  The path to the output directory is available as
  \lstinline|self.output_dir|.

  \+
  If \texttt{stdout} and \texttt{stderr}
  have been captured, the \textbf{relative} paths to the capture files
  are available as \lstinline|self.stdout| and
  \lstinline|self.stderr|.
\end{frame}


\begin{frame}[fragile]
  \frametitle{Post-processing features, II}

  For example, the following code logs a warning message if the
  standard error output is non-empty:
\begin{python}
class MyApp(Application):
  # ...
  def terminated(self):
    stderr_file = self.output_dir+"/"+self.stderr
    stderr_size = os.stat(stderr_file).st_size
    if stderr_size > 0:
      gc3libs.log.warn(
        "Application %s reported errors!", self)
\end{python}
\end{frame}


\begin{frame}[fragile]
  \frametitle{Useful in post-processing}\small

  These attributes are available in the \texttt{terminated()} method:

  \+
  \begin{describe}{\lstinline|self.inputs|}
    Python dictionary, mapping local (absolute) paths to remote paths (relative
    to execution directory)
  \end{describe}

  \+
  \begin{describe}{\lstinline|self.outputs|}
    Python dictionary, mapping remote paths (relative to execution directory) to
    \emph{URLs} where they have been copied. In particular,
    \lstinline|self.outputs.keys()| is the list of output file names.
  \end{describe}

  \+
  \begin{describe}{\lstinline|self.output_dir|}
    Path to the local directory where output files have been downloaded.
  \end{describe}
\end{frame}


\begin{frame}[fragile]
  \frametitle{Detour: How to $\ldots$ in Python}
  \small{}

  \begin{describe}{$\ldots$ list directory contents}
    Use the \lstinline|os.listdir()| function:
    \begin{python}
>>> import os
>>> os.listdir('/etc', 'passwd')
['journalctl', 'wdctl', 'uname', ...]
\end{python}
  Note that \lstinline|os.listdir()| returns a list of \emph{relative} file names.
  \end{describe}

  \begin{describe}{$\ldots$ concatenate a directory and a file name}
    Use the \lstinline|os.path.join()| function:
    \begin{python}
>>> import os
>>> os.path.join('/etc', 'passwd')
'/etc/passwd'
>>> os.path.isdir('/non/existent')
False
    \end{python}
  \end{describe}
\end{frame}


\begin{frame}[fragile]
  \frametitle{Detour: How to $\ldots$ in Python}
  \small{}

  \begin{describe}{$\ldots$ make a directory}
    Use the \lstinline|os.makedirs()| function:
    \begin{python}
>>> import os
>>> os.makedirs('pictures')
   \end{python}
   Note that \lstinline|os.makedirs()| will raise an error if the directory
   \emph{already exists}.
  \end{describe}

  \begin{describe}{$\ldots$ check that a directory exists}
    Use the \lstinline|os.path.isdir()| function:
    \begin{python}
>>> import os
>>> os.path.isdir('/tmp')
True
>>> os.path.isdir('/non/existent')
False
    \end{python}
  \end{describe}
\end{frame}


\begin{frame}
  \begin{exercise*}[6.A]

    In the \texttt{colorize.py} script from Exercise 4.A, modify the
    \texttt{ColorizeApp} application to move the output picture file
    into directory \texttt{/home/ubuntu/pictures}.

    \+ {\small (You might want to check out
      \url{http://stackoverflow.com/a/8858026/459543} if you're unsure
      how to move/rename a file with Python.)}
  \end{exercise*}
\end{frame}


\section{Termination status}
\part{Termination status}

\begin{frame}[fragile]
  \frametitle{A successful run or not?}

  There's a \emph{single TERMINATED state}, whatever the task outcome.
  You have to inspect the ``return code'' to determine the
  cause of ``task death''.

  \+
  Attribute \lstinline|.execution.returncode| provides a numeric termination
  status (with the same format and meaning as the POSIX termination
  status).

  \+
  The termination status combines two fields: the ``termination
  signal'' and the ``exit code''.

\end{frame}

\begin{frame}[fragile]
  \frametitle{Termination signal, I}

  The \texttt{.execution.signal} instance attribute is non-zero if
  the program was killed by a signal (e.g., memory error / segmentation fault).

  \+
  The \texttt{.execution.signal} instance attribute is zero only if
  the program run until termination. (\textbf{Beware!} This does not
  mean that it run \emph{correctly}: just that it halted by itself.)
\end{frame}


\begin{frame}[fragile]
  \frametitle{Termination signal, II}

  Read
  \href{http://man7.org/linux/man-pages/man7/signal.7.html}{\texttt{man
      7 signal}} for a list of OS signals and their numeric values.

  \+
  {\bfseries Note that GC3Pie uses some signal codes (not used
    by the OS) to represent its own specific errors.}

  \+
  For instance, if program \texttt{app} was cancelled by the user,
  \texttt{.execution.signal} will take the value 121:
\begin{python}
>>> print(app.execution.signal)
121
\end{python}

\begin{references}
  \tiny\url{https://github.com/uzh/gc3pie/blob/master/gc3libs/__init__.py#L1579}
\end{references}
\end{frame}


\begin{frame}[fragile]
  \frametitle{Exit code}

  The \texttt{.execution.exitcode} instance attribute holds the
  numeric exitcode of the executed command, or \texttt{None} if the
  command has not finished running yet.

  \+
  {\bfseries Note that the \texttt{.execution.exitcode} is guaranteed
    to have a valid value only if the \texttt{.execution.signal}
    attribute has the value 0.}

  \+
  The \texttt{.execution.exitcode} is the same exitcode that you
  would see when running a command directly in the terminal shell. (By
  convention, code 0 is successful termination, every other value
  indicates an error.)
\end{frame}


\begin{frame}
  \begin{exercise*}[6.B]

    Modify the grayscaling script \texttt{ex2c} (or the code it
    depends upon) so that, when a \texttt{GrayscaleApp} task has
    terminated execution, it prints:
    \begin{itemize}
    \item whether the program has been killed by a signal, and the signal number;
    \item whether the program has terminated by exiting, and the exit code.
    \end{itemize}
  \end{exercise*}
\end{frame}


\begin{frame}
  \begin{exercise*}[6.B+] \emph{(Bonus points)} Abstract the verbose
    \texttt{terminated} method from exercise 6.B into an application
    class \texttt{TermStatusApp}.

    \+
    Use Python class inheritance to add the \texttt{TermStatusApp}
    functionality into \texttt{GrayscaleApp}.
  \end{exercise*}
\end{frame}


\begin{frame}[fragile]\small
  \frametitle{Application-specific configuration}

  Application classes may be tagged so that parts of the configuration
  file can be overridden just for them.

  \+
  Suppose you tag the \texttt{GrayscaleApp} class by giving it this name:
\begin{python}
  class GrayscaleApp(Application):
    application_name = 'grayscale'
    # ~\itshape [\ldots]~
\end{python}
  then you can provide a specific VM image just for
  ``\texttt{grayscale}'' applications:
  \begin{stdout}
  # in the GC3Pie config file:
  [resource/sciencecloud]
  # ~\itshape [\ldots]~
  image_id=2b227d15-8f6a-42b0-b744-ede52ebe59f7
  grayscale_image_id=0cca5346-ca12-4cb4-8007-8875c10cce02
  \end{stdout}

  \+ Other configuration items that can be specialized are:
  \lstinline|instance_type|, \lstinline|user_data| (cloud),
  and \lstinline|prolog_file|, \lstinline|epilog_file|
  (batch-systems).
\end{frame}


\begin{frame}[fragile]
  \frametitle{Detour: How to $\ldots$ in Python}
  \small{}

  \begin{describe}{$\ldots$ take a substring of a string}
    Use the \texttt{text[\emph{start}:\emph{end}]} notation:
    \begin{python}
>>> text = 'awesome.m'
>>> text[0:-2]
'awesome'
    \end{python}
  \end{describe}

  \begin{describe}{$\ldots$ check for substring inclusion}
    Use the \lstinline|in| operator:
    \begin{python}
>>> 'awe' in 'awesome'
True
    \end{python}
  \end{describe}

  \begin{describe}{$\ldots$ read whole contents of a file}
    Use the \lstinline|read()| on a opened file:
    \begin{python}
>>> data = open('results.csv')
>>> data.read()
'50,50.123,...'
    \end{python}

  \end{describe}
\end{frame}


\begin{frame}[fragile]
  \begin{exercise*}[6.C] \emph{(Difficult)} \footnotesize

    MATLAB has the annoying habit of exiting with code 0 even when some error occurred.

    \+
    Write a \texttt{MatlabApp} application, which:
    \begin{itemize}
    \item is constructed by giving the path to a MATLAB `\texttt{.m}'
      script file, like this: \texttt{app = MatlabApp("\href{https://github.com/uzh/gc3pie/blob/master/docs/programmers/tutorials/workflows/downloads/ra.m}{ra.m}")};
    \item Runs the following command:
\begin{semiverbatim}
matlab -nodesktop -nojvm -r \emph{file}
\end{semiverbatim}
      where \emph{file.m} is the file given to the
      \texttt{MatlabApp()} constructor.
    \item captures the standard error output (\texttt{stderr}) of the MATLAB
      script and, if one of the strings ``\texttt{Out of memory.}'' or
      ``\texttt{exceeds maximum array size}'' occurs in it, sets the application
      exitcode to 11.
    \end{itemize}

    Verify that it works by running MATLAB script
    \href{https://github.com/uzh/gc3pie/blob/master/docs/programmers/tutorials/workflows/downloads/ra.m}{\texttt{ra.m}}
    many times over.  The script initializes a array of random size:
    for some values, the size exceeds the amount of available memory.
  \end{exercise*}
\end{frame}


\begin{frame}
  \frametitle{Global post-processing, I}
Further options for customizing a session-based script:

\+
\begin{describe}{\lstinline|before_main_loop(self)|}
  to execute some code \emph{before} the main loop starts.
\end{describe}

\+
\begin{describe}{\lstinline|after_main_loop(self)|}
  to execute some code \emph{after} the main loop, i.e., before the script
  quits. A list of all Application objects is available in the
  \lstinline|self.session.tasks.values()| list.
\end{describe}
\end{frame}


\begin{frame}[fragile]
  \frametitle{Global post-processing, II}
  Example: compute statistical distribution of termination statuses:

  \begin{python}
def after_main_loop(self):
  # check that all tasks are terminated
  can_postprocess = True
  for task in self.session.tasks.values():
    if task.execution.state != 'TERMINATED':
      can_postprocess = False
      break
  if can_postprocess:
    # do stuff... (see next slide)
  \end{python}
\end{frame}


\begin{frame}[fragile]
  \frametitle{Global post-processing, III}
  Example: compute statistical distribution of termination statuses (cont'd):

  \begin{python}
def after_main_loop(self):
  # ... (see prev slide)
  if can_postprocess:
    status_counts = defaultdict(int)
    for app in self.session.tasks.values():
      termstatus = app.execution.returncode
      status_counts[termstatus] += 1
  \end{python}

  \+\small Variable \lstinline|self.session.tasks| holds a mapping
  \lstinline|JobID|~$\Rightarrow$~\lstinline|Application|; thus
  \lstinline|self.session.tasks.values()| is a list of all the
  \texttt{Application} instances returned by \lstinline|new_tasks|
\end{frame}


\begin{frame}
  \frametitle{Detour: asset pricing, I}
  \small The script \texttt{simAsset.R} simulates asset pricing over a certain
  amount of time. Different pricing paths are generated using a
  \href{https://en.wikipedia.org/wiki/Wiener_process}{1D Brownian motion},
  all starting from the same initial price.
  \begin{center}
    \includegraphics[width=0.75\linewidth]{fig/simAsset.pdf}
  \end{center}
\end{frame}

\begin{frame}[fragile]
  \frametitle{Detour: asset pricing, II}
  \small
  You can run the \texttt{simAsset.R} script with these positional parameters:
  \begin{description}
  \item[$S_0$] stock price today (e.g., 50)
  \item[$\mu$] expected return (e.g., 0.04)
  \item[$\sigma$] volatility (e.g., 0.1)
  \item[$\delta$] size of time steps (e.g., 0.273)
  \item[$e$] days to expiry (e.g., 1000)
  \item[$N$] number of simulation paths to generate
  \end{description}

  For example:
\begin{semiverbatim}
  \$ Rscript simAsset.R 50 0.04 0.1 0.27 10 4
\end{semiverbatim}

  \+ Each run of \texttt{simAsset.R} produces two output files:
  \begin{description}
  \item[results.csv] table of generated data: each column is a simulation path, each row is a time step;
  \item[results.pdf] plot of the above.
  \end{description}
\end{frame}


\begin{frame}[fragile]
  \frametitle{Detour: How to $\ldots$ in Python}
  \small{}

  \begin{describe}{$\ldots$ compute the sum of a list of numbers}
    Use the built-in \lstinline|sum()| function:
    \begin{python}
>>> sum([1, 2, 3])
6
    \end{python}
  \end{describe}

  \begin{describe}{$\ldots$ get the number of items in a list}
    Use the built-in \lstinline|len()| function:
    \begin{python}
>>> len(['a', 'b','c'])
3
    \end{python}
  \end{describe}

  \begin{describe}{$\ldots$ convert a string to number}
    Use the built-in \lstinline|float()| or \lstinline|int()| functions:
    \begin{python}
>>> int('3')
3
    \end{python}
  \end{describe}
\end{frame}


\begin{frame}[fragile]
  \frametitle{Detour: How to $\ldots$ in Python}
  \small{}

  \begin{describe}{$\ldots$ read a CSV file}
    Use the \href{https://docs.python.org/2/library/csv.html}{\texttt{csv}}
    module from the standard library:
    \begin{python}
import csv

path = '/some/file.csv'
data = open(path)
rows = csv.reader(data)
for row in rows:
  # process row
    \end{python}
  \end{describe}

  \begin{describe}{$\ldots$ extract fields from CSV rows}
    When using the \href{https://docs.python.org/2/library/csv.html}{\texttt{csv}}
    module, rows are just tuples of values.  Use \texttt{row[\emph{i}]} to
    access the \texttt{\emph{i}}-th field in the row.
  \end{describe}
\end{frame}


\begin{frame}[fragile]
  \begin{exercise*}[6.D] \emph{(Difficult)}

    \+
    Write a \texttt{sim\_asset.py} program that:
    \begin{itemize}
    \item takes the same command-line positional arguments as
      \href{https://github.com/uzh/gc3pie/blob/master/docs/programmers/tutorials/workflows/downloads/simAsset.R}{\texttt{simAsset.R}},
      \emph{plus} an additional integer trailing parameter $P$;
    \item runs \texttt{simAsset.R} (in parallel) $P$ times with the given arguments (so, effectively simulates $N \cdot P$ price paths);
    \item reads all the generated \texttt{results.csv} files, and
    \item computes and prints the average value of the asset at the end of the simulated time, across all $N \cdot P$ price paths.
    \end{itemize}

    \+ {\footnotesize (For easier reading CSV files, you can use the standard
      \href{https://docs.python.org/2/library/csv.html}{\texttt{csv}}
      Python module,
      see:~\url{https://docs.python.org/2/library/csv.html})}
  \end{exercise*}
\end{frame}


\end{document}

%%% Local Variables:
%%% mode: latex
%%% TeX-master: t
%%% End:
